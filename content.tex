% <wc:start description="Content" max=800>

Attention Harvard Students! Periodically we publish actions the University
has taken to help us move efficiently and safely around campus. Here is our
latest bulletin.

Given that John Harvard invented the combustion engine, to honor his memory,
and in anticipation of great rejoicing among faculty and administrators, we
have recently elevated fuel-burning vehicles to our primary supported mode of
transport. All Harvard employees and contractors are henceforth encouraged to
move around our campus in whatever manner consumes the most fuel.

This has led to some changes you may have noticed. All paths wide enough to
accommodate vehicles, such as the ones ringing the Old Yard, are now roads
and open to traffic. Pedestrians are reminded to yield to all passing
traffic, especially trucks that idle menacingly right behind you while you
are strolling and enjoying a conversation. These vehicles are probably on a
super-important mission. If you do happen to walk through Harvard Yard
without noticing at least half-a-dozen vehicles, please cherish this special
moment. And then immediately call 5-1212 since it is possible that a zombie
invasion is taking place and you are in grave danger.

To further facilitate vehicular traffic we are revisiting campus traffic
signage. As part of our new ``Green is the new Red'' initiative we are
eliminating most stop signs, stop lights and pedestrian crosswalks. Stop
signs and lights were generally ignored by drivers and crosswalks seemed to
generate confusion, so this seems like an overdue simplification.
Additionally, at intersections where stop signs have been missing, like the
corner of Mill and Plympton Streets, Cambridge commuters have been trying to
hit students for years with limited success. Apparently College students have
developed excellent car-dodging reflexes and we trust that our new policies
will allow you to use them, well, pretty much everywhere. (For those of you
new to campus we suggest crossing in groups large enough to sway the moral
calculus of rapidly oncoming drivers. Twenty is a good starting point.)
Finally, the walk light at the corner of JFK and Memorial drive will be
shortened in duration from 6 seconds to 2, and will only appear twice an
hour. We trust that this will not inconvenience the athletes braving that
intersection daily, since you're all quick-moving and agile already.

Finally, you can't have cars without parking. Parking parking parking. We
love parking! Parking is great because it allows people who drive to campus
to stash their car nearby. We know most of you students don't drive. Driving
is a pleasure that's really only appreciated with time, and once you
appreciate it you'll understand the importance of parking. PARKING! Anyway,
we're just going to have to pull rank on this one. We know you like things
like Yards and Commons and such, for gathering or lounging or whatever it is
you non-drivers do when you're not spending hours enjoying Boston gridlock,
but really, you have plenty of places to do those things! Look, we promise
not to park in the dining hall, OK? But we're going to park everywhere else.
Forget Frisbee, colorful chairs and those quaint white plastic barriers. Just
imagine how gleeful the Yard will look shining with chrome! We're going to
replace the guard at the Harvard gate with a guy waiving a checkered flag:
just \$10 to park in historic Harvard Yard! (Take that, renegade tour guys!)
Upset? Don't push us. We remember exactly how to get out of that whole owning
Widener library deal, since you know what the City of Cambridge would put
there. More. Parking!

Whew, we're a bit out of breath from all this car talk, so let's briefly
address that unimportant minority of you who get around some other way. We
are stepping up our efforts to discourage bike usage on campus since we all
know that bikes---unlike cars---are dangerous. Bikers themselves confound the
problem with their refusal to learn the most trivial rules of the road.
Instead of being deterred by the maze of one-way streets and missing bike
lanes we have carefully constructude, you seem wedded to the strange idea
that biking should actually be a faster way of getting around campus. Wrong.
Only cars can do that!  So we'll make this real simple: show proper respect
to vehicles powered by polluting fossil fuels, or else. Pedestrians we're
together on, so we don't mind if you jump lights, ride bikes without brakes,
ride bikes you cannot stop at breakneck speeds, or blast through crowded
crosswalks. Just don't ever make us slow down. And helmets? Completely
optional! The fewer of you clogging our streets with your slow, clean,
sustainable pedaling, the better.

Even though we've failed to address one or two esoteric forms of transport
used by like, a few weirdos living in the Co-op, we're running out of time so
let's recap.  We all live and work together in a dense urban environment
where there is never enough parking. Also polar icecaps are melting. What
that all means is that cars are increasingly important, since we will need
them to drive inland once the water starts rising. Of almost equal importance
is parking, or places to put our cars in the increasingly rare moments we are
not using them, and of course to have them nearby in case this whole global
catastrophe stuff hits while we're at work. But our dedicated task force is
responding to this issue, and we can guarantee that Harvard's campus will be
completely car-friendly by 2017.

% <wc:end>

\textit{Geoffrey Challen '02--'03 is a Resident Tutor at Eliot House. The
views expressed are his and do not reflect official Harvard College policy.}
